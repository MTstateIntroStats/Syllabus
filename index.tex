% Options for packages loaded elsewhere
\PassOptionsToPackage{unicode}{hyperref}
\PassOptionsToPackage{hyphens}{url}
%
\documentclass[
]{article}
\usepackage{amsmath,amssymb}
\usepackage{iftex}
\ifPDFTeX
  \usepackage[T1]{fontenc}
  \usepackage[utf8]{inputenc}
  \usepackage{textcomp} % provide euro and other symbols
\else % if luatex or xetex
  \usepackage{unicode-math} % this also loads fontspec
  \defaultfontfeatures{Scale=MatchLowercase}
  \defaultfontfeatures[\rmfamily]{Ligatures=TeX,Scale=1}
\fi
\usepackage{lmodern}
\ifPDFTeX\else
  % xetex/luatex font selection
\fi
% Use upquote if available, for straight quotes in verbatim environments
\IfFileExists{upquote.sty}{\usepackage{upquote}}{}
\IfFileExists{microtype.sty}{% use microtype if available
  \usepackage[]{microtype}
  \UseMicrotypeSet[protrusion]{basicmath} % disable protrusion for tt fonts
}{}
\makeatletter
\@ifundefined{KOMAClassName}{% if non-KOMA class
  \IfFileExists{parskip.sty}{%
    \usepackage{parskip}
  }{% else
    \setlength{\parindent}{0pt}
    \setlength{\parskip}{6pt plus 2pt minus 1pt}}
}{% if KOMA class
  \KOMAoptions{parskip=half}}
\makeatother
\usepackage{xcolor}
\usepackage[margin=1in]{geometry}
\usepackage{longtable,booktabs,array}
\usepackage{calc} % for calculating minipage widths
% Correct order of tables after \paragraph or \subparagraph
\usepackage{etoolbox}
\makeatletter
\patchcmd\longtable{\par}{\if@noskipsec\mbox{}\fi\par}{}{}
\makeatother
% Allow footnotes in longtable head/foot
\IfFileExists{footnotehyper.sty}{\usepackage{footnotehyper}}{\usepackage{footnote}}
\makesavenoteenv{longtable}
\usepackage{graphicx}
\makeatletter
\def\maxwidth{\ifdim\Gin@nat@width>\linewidth\linewidth\else\Gin@nat@width\fi}
\def\maxheight{\ifdim\Gin@nat@height>\textheight\textheight\else\Gin@nat@height\fi}
\makeatother
% Scale images if necessary, so that they will not overflow the page
% margins by default, and it is still possible to overwrite the defaults
% using explicit options in \includegraphics[width, height, ...]{}
\setkeys{Gin}{width=\maxwidth,height=\maxheight,keepaspectratio}
% Set default figure placement to htbp
\makeatletter
\def\fps@figure{htbp}
\makeatother
\setlength{\emergencystretch}{3em} % prevent overfull lines
\providecommand{\tightlist}{%
  \setlength{\itemsep}{0pt}\setlength{\parskip}{0pt}}
\setcounter{secnumdepth}{-\maxdimen} % remove section numbering
\ifLuaTeX
  \usepackage{selnolig}  % disable illegal ligatures
\fi
\usepackage{bookmark}
\IfFileExists{xurl.sty}{\usepackage{xurl}}{} % add URL line breaks if available
\urlstyle{same}
\hypersetup{
  pdftitle={STAT 216: Introduction to Statistics},
  pdfauthor={Spring 2025 Syllabus},
  hidelinks,
  pdfcreator={LaTeX via pandoc}}

\title{STAT 216: Introduction to Statistics}
\author{Spring 2025 Syllabus}
\date{}

\begin{document}
\maketitle

{
\setcounter{tocdepth}{3}
\tableofcontents
}
\section{Instructor contact
information}\label{instructor-contact-information}

Your primary contact in STAT 216 is your instructor. If you have
concerns that cannot be answered by your instructor, you may reach out
to the Student Success Coordinator or the Course Supervisor.

Refer to your section's \textbf{Instructor Contact Information under D2L
Content} for your instructor and co-instructor/TA contact information.

\subsection{Student Success
Coordinator}\label{student-success-coordinator}

\href{http://www.math.montana.edu/directory/faculty/1524571/jade-schmidt}{\textbf{Jade
Schmidt}}\\
email:
\href{mailto:jade.schmidt2@montana.edu}{\nolinkurl{jade.schmidt2@montana.edu}}\\
Office: Wilson 2-263\\
Phone: (406) 994-6557

\begin{center}\rule{0.5\linewidth}{0.5pt}\end{center}

\subsection{Assistant Coordinator}\label{assistant-coordinator}

\href{http://www.math.montana.edu/directory/faculty/1582830/melinda-yager}{\textbf{Melinda
Yager}}\\
email:
\href{mailto:melinda.yager@montana.edu}{\nolinkurl{melinda.yager@montana.edu}}\\
Office: Wilson 2-260

\begin{center}\rule{0.5\linewidth}{0.5pt}\end{center}

\subsection{Course Supervisor}\label{course-supervisor}

\href{https://math.montana.edu/directory/faculty/1941032/stacey-hancock}{\textbf{Dr.~Stacey
Hancock}}\\
email:
\href{mailto:stacey.hancock@montana.edu}{\nolinkurl{stacey.hancock@montana.edu}}\\
Office: Wilson 2-195\\
Phone: (406) 994-5350

\begin{center}\rule{0.5\linewidth}{0.5pt}\end{center}

\section{Course calendars}\label{course-calendars}

\begin{itemize}
\tightlist
\item
  \href{calendars/S25-Stat216_Calendar.pdf}{STAT 216 calendar}
\item
  \href{http://catalog.montana.edu/academiccalendar/}{MSU academic
  calendar}
\item
  \href{https://www.montana.edu/registrar/Schedules.html}{MSU finals
  week schedule} (be sure to view the \textbf{Final Week Common
  (Combined) Section Groupings})
\end{itemize}

\begin{center}\rule{0.5\linewidth}{0.5pt}\end{center}

\section{Course description}\label{course-description}

Stat 216 is designed to engage you in the statistical investigation
process from developing a research question and data collection methods
to analyzing and communicating results. This course introduces basic
descriptive and inferential statistics using both traditional (normal
and \(t\)-distribution) and simulation approaches including confidence
intervals and hypothesis testing on means (one-sample, two-sample,
paired), proportions (one-sample, two-sample), regression and
correlation. You will be exposed to numerous examples of real-world
applications of statistics that are designed to help you develop a
conceptual understanding of statistics. After taking this course, you
should be able to:

\begin{itemize}
\tightlist
\item
  Understand and appreciate how statistics affects your daily life and
  the fundamental role of statistics in all disciplines.
\item
  Evaluate statistics and statistical studies you encounter in your
  other courses.
\item
  Critically read news stories based on statistical studies as an
  informed consumer of data.
\item
  Assess the role of randomness and variability in different contexts.
\item
  Use basic methods to conduct and analyze statistical studies using
  statistical software.
\item
  Evaluate and communicate answers to the four pillars of statistical
  inference: How strong is the evidence of an effect? What is the size
  of the effect? How broadly do the conclusions apply? Can we say what
  caused the observed difference?
\end{itemize}

\subsubsection{MUS STAT 216 learning
outcomes}\label{mus-stat-216-learning-outcomes}

\begin{enumerate}
\def\labelenumi{\arabic{enumi}.}
\tightlist
\item
  Understand how to describe the characteristics of a distribution.
\item
  Understand how data can be collected, and how data collection dictates
  the choice of statistical method and appropriate statistical
  inference.
\item
  Interpret and communicate the outcomes of estimation and hypothesis
  tests in the context of a problem.
\item
  To understand the scope of inference for a given dataset.
\end{enumerate}

\subsubsection{CORE 2.0}\label{core-2.0}

This course fulfills the Quantitative Reasoning (Q) CORE 2.0 requirement
because learning probability and statistics allows us to disentangle
what's really happening in nature from ``noise'' inherent in data
collection. It allows us to evaluate claims from advertisements and
results of polls and builds critical thinking skills which form the
basis of statistical inference. Students completing a Core 2.0
Quantitative Reasoning (Q) course should demonstrate an ability to:

\begin{enumerate}
\def\labelenumi{\arabic{enumi}.}
\tightlist
\item
  Interpret and draw inferences from mathematical models such as
  formulas, graphs, diagrams or tables.
\item
  Represent mathematical information numerically, symbolically and
  visually.
\item
  Employ quantitative methods in symbolic systems such as, arithmetic,
  algebra, or geometry to solve problems.
\end{enumerate}

\begin{center}\rule{0.5\linewidth}{0.5pt}\end{center}

\section{Prerequisites}\label{prerequisites}

Entrance to STAT 216 requires at least one of the following be met:

\begin{itemize}
\tightlist
\item
  Grade of C- or better in a 100-level math course (or equivalent)
\item
  Grade of B or better M090 or the M063/M090 co-requisite
\item
  Level 30 on the \href{http://www.montana.edu/testing/MPLEX.html}{Math
  Placement Exam} or a combination of a good score on Math portion of
  SAT (540 or higher) or ACT (23 or higher) and/or good high school GPA

  \begin{itemize}
  \tightlist
  \item
    See the
    \href{http://www.math.montana.edu/undergrad/documents/MHiearchyFlowchart.pdf}{Math
    Prerequisite Flowchart} for more details.
  \end{itemize}
\end{itemize}

You should have familiarity with computers and technology (e.g.,
Internet browsing, word processing, opening/saving files, converting
files to PDF format, sending and receiving e-mail, etc.).

\begin{center}\rule{0.5\linewidth}{0.5pt}\end{center}

\section{Course materials and
resources}\label{course-materials-and-resources}

\subsubsection{Online textbook and
coursepack}\label{online-textbook-and-coursepack}

Two ``textbooks'' are required for this course:

\begin{enumerate}
\def\labelenumi{\arabic{enumi}.}
\tightlist
\item
  \href{https://mtstateintrostats.github.io/IntroStatTextbook/}{\emph{Montana
  State Introductory Statistics with R}} --- our free, online textbook
\item
  \emph{STAT 216 Coursepack} --- workbook with key topics, video notes,
  in-class activities, and labs
\end{enumerate}

The \emph{Stat 216 Coursepack} of in-class activities is available for
purchase in the \href{https://www.msubookstore.org/}{MSU Bookstore}. You
may purchase the coursepack in person, or you may purchase online and
have the coursepack shipped to you. Students are expected to bring the
coursepack to class each day and to complete the activities within the
coursepack.

\subsubsection{RStudio}\label{rstudio}

We will be using the statistical software
\href{https://www.r-project.org/}{R} through the IDE
\href{https://rstudio.com/products/rstudio/}{RStudio} for data
visualization and statistical analyses.

You will access this software through the MSU RStudio server:
\href{https://rstudio.math.montana.edu/}{rstudio.math.montana.edu}. Your
username is your 7-character NetID (in the form x\#\#x\#\#\#, where x is
a letter and \# is a number), and your password is the password
associated with your NetID. Your email address will not work to log in
to the RStudio server.

\begin{itemize}
\tightlist
\item
  Please note: Your netID password expires every 6 months. It is HIGHLY
  recommended that you reset your netID password BEFORE attempting to
  login to the RStudio server. You can reset your netID password in the
  \href{https://pwreset.montana.edu/react/}{MSU password portal}.
\end{itemize}

See the
\href{https://mtstateintrostats.github.io/IntroStatTextbook/rstudio.html\#alternative-options-for-accessing-rstudio}{Statistical
Computing} section in the Welcome chapter of our textbook for
alternative options for accessing RStudio.

\subsubsection{Required course software}\label{required-course-software}

All students are required to have a word processor and spreadsheet
software installed on the personal device they plan to use for this
course. We \emph{highly} recommend the use of Word and Excel. If you do
not currently have Word and/or Excel installed on your device, you can
download the Microsoft Office 365 for Students for free by following the
instructions
\href{https://coe.montana.edu/it/students/student-software.html}{here}

\subsubsection{Learning management
tools}\label{learning-management-tools}

\begin{itemize}
\item
  \href{https://ecat1.montana.edu/}{\textbf{D2L}}: Find your instructor
  and co-instructor/TA contact info, announcements, exploration
  information, instructor notes, exam review material, assignment and
  data files, discussion forums, gradebook.

  \begin{itemize}
  \tightlist
  \item
    \emph{Important}: Make sure you are receiving email notifications
    for any D2L activity. In D2L, click on your name, then
    Notifications. Check that D2L is using an email address that you
    regularly check; you have the option of registering a mobile number.
    Check the boxes to get notifications for announcements, content,
    discussions, and grades.
  \item
    If you have a question about the course materials, computing, or
    logistics, please post your question to your D2L discussion board
    instead of emailing your instructor. This ensures all students can
    benefit from the responses. Other students are encouraged to
    respond.
  \end{itemize}
\item
  \href{https://www.gradescope.com/}{\textbf{Gradescope}}: Submit and
  review quizzes and assignments, review exam grades. For more details,
  see the provided Stat 216 FAQs document on D2L Content
  --\textgreater{} Primary Resources.
\item
  \href{https://math.montana.edu/undergrad/msc/index.html}{\textbf{Mathematics
  and Statistics Center}}: Free drop-in tutoring for 100- and 200-level
  math and stat courses in Romney Hall 220.
\end{itemize}

\begin{center}\rule{0.5\linewidth}{0.5pt}\end{center}

\section{Course format and
organization}\label{course-format-and-organization}

Stat 216 will meet 3 times per week.

\begin{itemize}
\tightlist
\item
  During class meetings, students will \textbf{complete} activities and
  labs in assigned groups with the assistance of the teaching team.
\item
  Per the \href{calendars/S25-Stat216_Calendar.pdf}{STAT 216 calendar},
  the course will be broken down into modules and units.
\item
  To begin each module, students will

  \begin{itemize}
  \tightlist
  \item
    \textbf{read} assigned sections of the online textbook on that
    topic's content.
  \item
    \textbf{watch} assigned videos on D2L and complete the video notes
    located in the Stat 216 coursepack \textbf{(video note completion is
    checked in class)}
  \item
    complete \textbf{video/reading} quizzes on
    \href{https://www.gradescope.com/}{Gradescope} \textbf{before} their
    class meets.
  \end{itemize}
\item
  \textbf{Activities} will take place during class in assigned teams
  with the assistance of the instruction team. Select problems from the
  activity will be turned into Gradescope before class ends.
\item
  After all activities within a module have been completed, students
  will complete a \textbf{Module Lab} during class time. The lab will
  serve as a review of key concepts from the module. Select problems
  from the lab will be turned into Gradescope before class ends.
\item
  After all modules within a unit are complete, students will complete a
  \textbf{Midterm Exam} over the topics covered.
\item
  Students will also complete one \textbf{assignment} in
  \href{https://www.gradescope.com/}{Gradescope} per week, typically due
  on Mondays.
\end{itemize}

\begin{center}\rule{0.5\linewidth}{0.5pt}\end{center}

\section{Course assessment}\label{course-assessment}

Your grade in STAT 216 will contain the following components.

\begin{center}\includegraphics[width=0.75\linewidth]{index_files/figure-latex/unnamed-chunk-1-1} \end{center}

\subsubsection{Preparation (10\%)}\label{preparation-10}

To begin each module, you will be expected to complete the assigned
textbook reading, watch the assigned videos in D2L and take complete
guided notes on the videos prior to class. Guided video notes are found
in the coursepack and will be checked off at the start of class the
first day of the new module. You will need to complete video/reading
quizzes, which are found on Gradescope and due by \textbf{the start of
class}. Up until your class time, you can retake the video/reading
quizzes as many times as you like by clicking Resubmit in Gradescope to
open and edit any question answer.

\begin{itemize}
\tightlist
\item
  Video/reading quizzes are due \textbf{before your class starts on
  dates indicated on the course calendar}.
\item
  Guided video notes, found in the coursepack purchased from the MSU
  Bookstore, will be \textbf{checked for completion at the start of
  class the first day of a new module}.
\item
  The lowest (1) video quiz and the lowest (1) video note grade will be
  dropped at the end of the semester.
\end{itemize}

\subsubsection{Engagement (10\%)}\label{engagement-10}

Every class day, you will meet with your classmates and instructor team
to work through that day's coursepack group activity. Attendance and
completion of the in-class activities and exit tickets (reviewing select
questions from the activity) counts towards this portion of your grade.

\begin{itemize}
\tightlist
\item
  Activities must be completed in the Stat 216 Coursepack. \textbf{If
  you prefer to complete the activity on a pdf copy using a
  stylus-enabled device, please speak to your instructor, and he or she
  will provide you with a PDF copy of the coursepack.}
\item
  As a group, you will submit an `exit ticket' (answers to select
  questions from the activity) to \textbf{Gradescope at the end of each
  class period}.
\item
  Activities will be checked for completion at the \textbf{beginning of
  the following class period}.
\item
  Policy for missed activities:

  \begin{itemize}
  \tightlist
  \item
    We will not record or post lectures/asynchronous learning
    opportunities.
  \item
    Students get a ``free pass'' for up to three activities per Unit, no
    questions asked, \emph{but students are expected to communicate any
    absences with their section instructor}.
    Illnesses/emergencies/school-related absences are included in these
    three; if students have extraneous circumstances, they are
    encouraged to talk to their instructor.
  \item
    For illnesses or when students cannot attend class, we recommend
    that the student video conference into class with their group.
    Students attending class remotely can show their activity via video
    conference for credit. Video conferencing should be set up by
    students' teammates. If the student is uncomfortable asking, the
    instructor can facilitate that conversation (e.g., email the
    students' teammates, cc'ing the student, and ask if anyone can setup
    a WebEx meeting for class).
  \item
    Students who are unable to attend class remotely AND e-mail their
    instructor \textbf{prior to class time} regarding an absence will
    receive a 0 for the class activity check but will be allowed a
    24-hour extension to complete the exit ticket.
  \end{itemize}
\item
  The three (3) lowest activity/lab completion check grades and the
  lowest (1) exit ticket grade EACH unit will be dropped.
\end{itemize}

\subsubsection{Module Labs (10\%)}\label{module-labs-10}

After a module has been completed, you will meet with your classmates
and instructor team to work through a RStudio group lab, which is
provided in your coursepack and will be completed during class time. The
lab will reinforce the topics learned in the activities but with the use
of RStudio for exploring and analyzing data.

\begin{itemize}
\tightlist
\item
  Each group will turn in selected questions from the lab to Gradescope.
  Labs are due \textbf{at 9 pm Mountain Time on dates indicated on the
  course calendar}.
\item
  The lowest (1) lab grade will be dropped at the end of the semester.
\item
  Each student will also have the lab checked for completion at the
  \textbf{beginning of the following class period} for that class
  period's Engagement grade.
\end{itemize}

\subsubsection{Assignments (10\%)}\label{assignments-10}

You will complete weekly assignments in
\href{https://www.gradescope.com/}{Gradescope}. These should be
completed individually (meaning all answers should be written in your
own words), but you may use your classmates, tutors, or your
instructor/co-instructor/TA for assistance.

\begin{itemize}
\tightlist
\item
  Weekly assignments are due \textbf{Monday at 9pm Mountain Time each
  week, covering the previous week's content}.
\item
  The lowest (1) assignment grade will be dropped at the end of the
  semester.
\item
  Please review the discussion of the use of \hyperref[AI]{AI} on
  assignments.
\end{itemize}

\subsubsection{Midterm exams (30\%)}\label{midterm-exams-30}

There will be two midterm exams, each consisting two parts: a group
portion and an individual portion. Both portions will be taken in class
during your normal in-class time on subsequent days (Group exams on
Wednesdays, Individual exams on Fridays). A practice individual exam
will be released in D2L one week prior to the exam, with solutions to
the practice exam released in D2L the Sunday prior to the exam. Further
details, resources, and instructions for each exam will be posted the
week prior to the exam in D2L.

\paragraph{\texorpdfstring{\emph{Group
portion}:}{Group portion:}}\label{group-portion}

Group midterm exams will be \textbf{Wednesday February 12 and Wednesday
March 26 (during normal class time)}.

\begin{itemize}
\tightlist
\item
  The group portion of the midterm exams will be worth 20\% of your
  midterm exam grade.
\item
  The group midterm exams are open book, open resources provided by Stat
  216 instructors (anything posted to D2L or feedback found in
  Gradescope)
\item
  You will be allowed a calculator on the group midterm exams.
\item
  You \textbf{will} be required to use RStudio on the group midterm
  exams.
\item
  \textbf{If you miss more than 3 class days in a unit, you must
  complete the group midterm exam for that unit individually.}
\end{itemize}

\paragraph{\texorpdfstring{\emph{Individual
portion}:}{Individual portion:}}\label{individual-portion}

Individual midterm exams will be \textbf{Friday February 14 and Friday
March 28 (during normal class time)}.

\begin{itemize}
\tightlist
\item
  The individual portion of the midterm exams will be worth 80\% of your
  midterm exam grade.
\item
  The individual midterm exams are closed book, closed note.
\item
  You will be allowed to create a one page note sheet for each
  individual exam. You will also be provided a one page formula sheet
  during the exam.
\item
  You will be allowed a calculator on the individual midterm exam.
\item
  You will \textbf{not} be required to use RStudio on the individual
  midterm exams.
\end{itemize}

\subsubsection{Final exam (30\%)}\label{final-exam-30}

The final exam will consist of two parts: a group portion and an
individual portion. The group final exam will be taken in class during
your normal in-class time on the final Wednesday and Friday of classes
(prior to Finals week). A practice individual exam will be released in
D2L one week prior to the exam, with solutions to the practice exam
released in D2L the Sunday prior to the exam. Further details,
resources, and instructions for each exam will be posted the week prior
to the exam in D2L.

The individual final exam is a common hour exam with the date/time set
by the Registrar. \textbf{Understand that attending that common hour
exam is part of your commitment when you enroll in the course.}

\paragraph{\texorpdfstring{\emph{Group
portion}:}{Group portion:}}\label{group-portion-1}

Group final exam will be \textbf{Wednesday April 30 and Friday May 2
(during normal class time)}.

\begin{itemize}
\tightlist
\item
  The group portion of the final exam will be worth 20\% of your final
  exam grade.
\item
  The group final exam is open book, open notes.
\item
  You will be allowed a calculator on the group final exam.
\item
  You \textbf{will} be required to use RStudio on the group final exam.
\item
  \textbf{If you miss more than 3 class days in Unit 3, you must
  complete the group final exam individually.}
\end{itemize}

\paragraph{\texorpdfstring{\emph{Individual
portion}:}{Individual portion:}}\label{individual-portion-1}

Individual common hour final exam will be \textbf{Tuesday, May 6th, from
12:00 - 1:50 pm}.

\begin{itemize}
\tightlist
\item
  The individual portion of the final exam will be worth 80\% of your
  final exam grade.
\item
  No potential final exam questions will be released.
\item
  The individual final exam is closed book.
\item
  You will be allowed to create a one page note sheet for the exam. You
  will also be provided a one page formula sheet during the exam.
\item
  You will be allowed a calculator on the individual final exam.
\item
  You will \textbf{not} be required to use RStudio on the individual
  final exams.
\item
  Understand that attending that common hour exam is part of your
  commitment when you enroll in the course.
\end{itemize}

\subsubsection{Letter grades}\label{letter-grades}

Final course grades will be determined according to the following scale.

\begin{longtable}[]{@{}ll@{}}
\toprule\noalign{}
Letter Grade & Weighted Score \\
\midrule\noalign{}
\endhead
\bottomrule\noalign{}
\endlastfoot
A & 93-100\% \\
A- & 90-92\% \\
B+ & 87-89\% \\
B & 83-86\% \\
B- & 80-82\% \\
C+ & 77-79\% \\
C & 70-76\% \\
D & 60-69\% \\
F & \textless59\% \\
\end{longtable}

The grade cutoffs may be shifted \emph{downward} at the end of the
semester based on student performance (never upward).

\begin{center}\rule{0.5\linewidth}{0.5pt}\end{center}

\section{Late work policies}\label{late-work-policies}

Note: we \textbf{highly} recommend saving your answers for each question
while you complete all work in Gradescope. This will ensure you can
return to labs, video/reading quizzes, exit tickets, or assignments at a
later date without fear of losing any progress. Additionally, Gradescope
will automatically submit any saved work when the due date passes,
ensuring you earn up to full credit for all problems completed on time.

\begin{itemize}
\tightlist
\item
  \textbf{Video/Reading Quizzes and Assignments}:

  \begin{itemize}
  \tightlist
  \item
    You may take the video/reading quizzes or assignments in Gradescope
    as many times as you like up until the due date using the Resubmit
    button to re-open a quiz.
  \item
    Extensions on these quizzes are not given unless extenuating
    circumstances are present which are communicated to the Student
    Success Coordinator, \href{jade.schmidt2@montana.edu}{Jade Schmidt}.
  \item
    For assignments, while the due dates are 9 pm on Mondays, you will
    be allowed to turn in work until 11:59 pm on the due date.
    Assignment submissions that are received between 9 pm and 11:59 pm
    will receive a 5\% grade deduction.
  \end{itemize}
\item
  \textbf{Exit Tickets and Labs}:

  \begin{itemize}
  \tightlist
  \item
    You may take the exit tickets and labs in Gradescope as many times
    as you like up until the due date using the Resubmit button to
    re-open an assessment.
  \item
    If you miss class and email your section instructor about your
    absence \textbf{prior to class time}, you will be given a 24-hour
    extension to complete the exit ticket or lab.
  \item
    If possible, it is highly recommended to attend class virtually (via
    video conferencing with your group) to complete the lab or exit
    ticket with your team.\\
  \item
    Further extensions on exit tickets or labs, or when an absence is
    not communicated prior to class time, are not given unless
    extenuating circumstances are present. Please email the Student
    Success Coordinator, \href{jade.schmidt2@montana.edu}{Jade Schmidt}
    if you feel your missed exit ticket or lab falls into this category.
  \item
    For labs, while the due dates are 9 pm on the day the lab is
    completed in class, you will be allowed to turn in work until 11:59
    pm on the due date. Lab submissions that are received between 9 pm
    and 11:59 pm will receive a 5\% grade deduction.
  \end{itemize}
\item
  \textbf{Activities}: Attendance in this course is critical for success
  and is therefore required. The in-class activity, exit tickets, and
  lab grades are a proxy for attendance and engagement. Students are
  expected to be in class during in-class activities and labs to provide
  support to each other and their teammates while working through the
  material. We will not record or post lectures/asynchronous learning
  opportunities. Students get a ``free pass'' for up to three class days
  per Unit, no questions asked, but are requested to communicate with
  their section instructor anytime they miss class.
  Illnesses/emergencies/school-related absences are included in these
  three; if students have extraneous circumstances, they are encouraged
  to talk to their instructor.

  \begin{itemize}
  \tightlist
  \item
    For illnesses or when students cannot attend class, we recommend
    that the student video conference into class with their group
    (preferably using WebEx or Zoom). Students attending class remotely
    can show their activity and participate in the exit ticket over
    video conference for credit. Video conferencing should be set up by
    students' teammates. If the student is uncomfortable asking, the
    instructor can facilitate that conversation (e.g., email the
    students' teammates, cc'ing the student, and ask if anyone can setup
    a WebEx meeting for class).
  \end{itemize}
\item
  \textbf{Exams}:

  \begin{itemize}
  \tightlist
  \item
    Students that are in quarantine but healthy enough to take the exam
    should email Student Success Coordinator
    \href{mailto:\%20jade.schmidt2@montana.edu}{Jade Schmidt} to make
    alternative arrangements. The group exams may be taken remotely and
    proctored via WebEx but all individual exams must be taken in
    person.
  \item
    If you are ill to the point of not being able to take the exam,
    please email Student Success Coordinator
    \href{mailto:\%20jade.schmidt2@montana.edu}{Jade Schmidt} to make
    alternative arrangements.\\
  \item
    Students who miss the exam without contacting the instructor prior
    to the exam will receive a zero on the exam.
  \item
    Work is not a legitimate reason for an exam absence.
  \end{itemize}
\end{itemize}

\begin{center}\rule{0.5\linewidth}{0.5pt}\end{center}

\section{COVID-19 policies and health-related class
absences}\label{covid-19-policies-and-health-related-class-absences}

Face masks are recommended, but not required, for students, faculty and
staff indoors on campus.

Please evaluate your own health status regularly and refrain from
attending class and other on-campus events if you are ill.~MSU students
who miss class due to illness will be given opportunities to access
course materials online. You are encouraged to seek appropriate medical
attention for treatment of illness. In the event of contagious illness,
please do not come to class or to campus to turn in work. Instead notify
us by email about your absence as soon as practical, so that
accommodations can be made. Please note that documentation (a Doctor's
note) for medical excuses is not required. MSU University Health
Partners--as part their commitment to maintain patient confidentiality,
to encourage more appropriate use of healthcare resources, and to
support meaningful dialogue between instructors and students--does not
provide such documentation.

\begin{center}\rule{0.5\linewidth}{0.5pt}\end{center}

\section{Diversity and inclusivity}\label{diversity-and-inclusivity}

Respect for Diversity: It is our intent that students from all diverse
backgrounds and perspectives be well-served by this course, that
students' learning needs be addressed both in and out of class, and that
the diversity that students bring to this class be viewed as a resource,
strength and benefit. It is our intent to present materials and
activities that are respectful of diversity: gender identity, sexual
orientation, disability, age, socioeconomic status, ethnicity, race,
religion, culture, perspective, and other background characteristics.
Your suggestions about how to improve the value of diversity in this
course are encouraged and appreciated. Please let us know ways to
improve the effectiveness of the course for you personally or for other
students or student groups.

In addition, in scheduling exams, we have attempted to avoid conflicts
with major religious holidays. If, however, we have inadvertently
scheduled an exam or major deadline that creates a conflict with your
religious observances, please let us know as soon as possible so that we
can make other arrangements.

Support for Inclusivity: We support an inclusive learning environment
where diversity and individual differences are understood, respected,
appreciated, and recognized as a source of strength. We expect that
students, faculty, administrators and staff at MSU will respect
differences and demonstrate diligence in understanding how other
peoples' perspectives, behaviors, and worldviews may be different from
their own.

\begin{center}\rule{0.5\linewidth}{0.5pt}\end{center}

\section{Policy on academic
misconduct}\label{policy-on-academic-misconduct}

Students in an academic setting are responsible for approaching all
assignments with rigor, integrity, and in compliance with the University
Code of Student Conduct. This responsibility includes:

\begin{enumerate}
\def\labelenumi{\arabic{enumi}.}
\tightlist
\item
  consulting and analyzing sources that are relevant to the topic of
  inquiry;
\item
  clearly acknowledging when they draw from the ideas or the phrasing of
  those sources in their own writing;
\item
  learning and using appropriate citation conventions within the field
  in which they are studying; and
\item
  asking their instructor for guidance when they are uncertain of how to
  acknowledge the contributions of others in their thinking and writing.
\end{enumerate}

When students fail to adhere to these responsibilities, they may
intentionally or unintentionally ``use someone else's language, ideas,
or other original (not common-knowledge) material without properly
acknowledging its source'' \url{http://www.wpacouncil.org}. When the act
is intentional, the student has engaged in plagiarism.

Plagiarism is an act of academic misconduct, which carries with it
consequences including, but not limited to, receiving a course grade of
``F'' and a report to the Office of the Dean of Students. Unfortunately,
it is not always clear if the misuse of sources is intentional or
unintentional, which means that you may be accused of plagiarism even if
you do not intentionally plagiarize. If you have any questions regarding
use and citation of sources in your academic writing, you are
responsible for consulting with your instructor before the assignment
due date. In addition, you can work with an MSU Writing Center tutor at
any point in your writing process, including when you are integrating or
citing sources. You can make an appointment and find citation resources
at www.montana.edu/writingcenter.

\textbf{ In STAT 216, assignments that include the same wording as
another student, regardless of whether that student was cited in your
sources, will be considered plagiarism and will be treated as such.
Students involved in plagiarism on assignments (all parties involved)
will receive a zero grade on that assignment. The second offense will
result in a zero on that assignment, and the incident will be reported
to the Dean of Students. Academic misconduct on an exam will result in a
zero on that exam and will be reported to the Dean of Students, without
exception.}

\href{https://www.montana.edu/deanofstudents/academicmisconduct/academicmisconduct.html}{More
information about Academic Misconduct from the Dean of Students}

\begin{center}\rule{0.5\linewidth}{0.5pt}\end{center}

\section{Policy on intellectual
property}\label{policy-on-intellectual-property}

This syllabus, course lectures and presentations, and any course
materials provided throughout this term are protected by U.S. copyright
laws. Students enrolled in the course may use them for their own
research and educational purposes. However, reproducing, selling or
otherwise distributing these materials without written permission of the
copyright owner is expressly prohibited, including providing materials
to commercial platforms such as Chegg or CourseHero. Doing so may
constitute a violation of U.S. copyright law as well as MSU's Code of
Student Conduct.

\begin{center}\rule{0.5\linewidth}{0.5pt}\end{center}

\section{Policy on the use of AI language models}\label{AI}

In this course, you may utilize generative AI language models, including
ChatGPT, as a resource to support your work outside of class (during
class, you should seek assistance from group members or instructors). AI
language models are powerful tools developed to generate text based on
the input provided. \textbf{While the AI language models can help refine
your writing and coding, it is important to remember that it is an AI
system and not a substitute for your critical thinking and creativity.
Due to the nature of statistics and this course, an AI-generated answer
may be incomplete, overly complex, or even incorrect.} If you do not
understand a concept or a question asked, we \emph{highly} recommend
visiting the \href{https://math.montana.edu/undergrad/msc/}{Math and
Stats Center}, emailing or visiting with a member of your instructional
team, or using the search feature within the online textbook before
turning to Google or AI.

If you choose to use this tool, apply it as a supplement and do not rely
solely on its suggestions. Ultimately, you are responsible for the
content and quality of your work. Therefore, you should critically
evaluate ChatGPT outputs for accuracy, potential bias, and relevancy.
When utilizing AI language models, it is essential to ensure that your
writing and coding remains original and properly attributed, including
citing outputs or text generated by ChatGPT. \textbf{If you choose to
use AI language models to assist you on labs or assignments, you must
cite the source used. Failure to do so will result in earning a 0 on all
problems in which AI language models usage has been detected.}

Please see the How to cite ChatGPT in MLA Style
\href{https://style.mla.org/citing-generative-ai/}{resource}. We
encourage you to use AI language models to enhance your writing and
coding skills, experiment with its capabilities, and learn from its
suggestions. If you have any questions or concerns regarding using AI
language models for assignments, please discuss them with us.

\end{document}
